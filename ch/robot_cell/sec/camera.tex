%!TEX root = ../../../main.tex
%%---------------------------------------------------------------------------
\section{Camera mounting and calibration}
\label{sec:rc_camera}
%%---------------------------------------------------------------------------
The provided camera is a \textit{Logitech webcam C930e}. The camera was mounted on the robot end effector. This was done from the hypothesis that a eye-in-hand camera/robot relationship was easier to calibrate compared to a hand to eye system and since there is no concerns with colliding with the camera the view can be closer to the grasping area.

Camera parameters was controlled using the video4Linux api and driver. A resolution of 1280x720 with a 30 FPS was chosen. The resolution was kept high in order to have a rich details in the image and therefore making detecting bricks more stable with a cost of increased computation time due to large image sizes. 
The auto focus of the camera was disabled and the focus was tuned manually since the camera will be fixed in place relative to the bricks.
Brightness, contrast and exposure was also manually tuned in order to capture images with minimum variance. This makes it easier to apply vision algorithms.

Camera calibration was done using the camera\_calibration ROS package. Using this tool the intrinsic parameters of the camera was calculated from images of a 9x7 chequerboard held in front of the camera in 73 different positions. Rectification of the image was then performed using the image\_proc ROS package.
