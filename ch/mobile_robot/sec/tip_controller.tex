%!TEX root = ../../../main.tex
\section{Tip controller} % (fold)
\label{sec:mr_tip_controller}

The purpose of tip controller is to create a simple and effective way of 
delivering a large amount of LEGO bricks to the conveyor belt autonomously. 
This is done by controlling the mounted belt drive with a stepper motor, an 
H-bridge and a Arduino microcontroller. To control the limits of tipping 
mechanism, two microswitches were added(Tip down/tip up).\\

\subsection{Stepper controller} % (fold)
\label{sub:stepper_controller}
The PK-244-01A stepper motor is bipolar. This means that it requires (at least) 
four wires to control the two coils in contrast to a unipolar motor which has 1 
supply wire and usually only requires two control wires. 
The four wires are driven by the L298N dual 
H-bridge, powered by 12V from on-board pc power supply. Four control pins are 
connected to the 
Arduino and signals on these controls the motor wires. Consulting datasheet for 
the motor, a minimum of 4V, 1.2A is required to drive the motor. Timing is 
calculated as a minimum of 
3ms delay between steps in the timing diagram. To get an initial high torque 
from the motor, a half-stepping sequence was chosen and a ramp function was 
created to slowly start the motor and exponentially increase speed when the 
tip is in full motion. Due to friction in the system and less torque the 
full-step sequence was disregarded. The Arduino software has a built-in library 
for stepper motors, but this was found difficult to manage in regards to delay 
of the steps. 
% subsection stepper_controller (end)

\subsection{Arduino and serial communication} % (fold)
\label{sub:arduino_and_serial_communication}
Setting a general serial communication to the Arduino is straight-forward. The 
speed is chosen with regards to the ROS node and a serial connection is 
initialized with the \textbf{Serial.begin(speed)} command. The Arduino is set 
up to 
receive a character "u" for tipping(up), and "d" for leveling(down). When an 
action has successfully executed, a "d" (done) is returned to the ROS node.
% subsection arduino_and_serial_communication (end)


% section tip_controller (end)
