%!TEX root = ../../main.tex
%----------------------------------------------------------------------------
\chapter{Conclusion}\label{chap:conclusion}
%----------------------------------------------------------------------------
Mobile robot\\
The tasks of the mobile platform was the transportation and loading of the LEGO bricks to be sorted. In transporting it would have to operate on two different kinds of navigation environments. One being guided by black lines and one in an unprepared area. The mobile robot system was able to successfully navigate in both these areas thus completing its transportation role. Though the execution during 24 hour test was not absolutely flawless it did perform reasonably well. A large contributor to navigation errors was traffic jam or collisions with other robot systems. 

The loading of bricks onto the conveyor was done by modifying the tipper on the original FrobitPro. After the modifications the transfer was achieved repeatedly  during the final test. 

All in all the mobile robot was able to go to the brick dispenser, transport bricks in the open area and along the guide lines to the conveyor where they were unloaded. It could then go to the robot to pick up the final order and transport it back to the charging station where it would stay and charge until battery level was sufficient for the next order.


%%% Local Variables:
%%% mode: latex
%%% TeX-master: "main"
%%% End: